\clearpage
\phantomsection% 
\addcontentsline{toc}{chapter}{Kata Pengantar}
\thispagestyle{fancy}

\begin{justifying}
	\large \bfseries \centering \MakeUppercase{Kata Pengantar}
	\par
	\normalsize \normalfont \justifying
	
	Segala puji dan syukur kepada Tuhan Yesus Kristus yang merupakan sumber hikmat dan pengharapan sejati, atas kasih karunia-Nya yang senantiasa menyertai setiap langkah penulis hingga akhirnya dapat menyelesaikan Tugas Akhir berjudul “Identifikasi Penyakit pada Daun Bibit Kelapa Sawit dengan Pendekatan \textit{Naïve Bayes} Menggunakan Optimasi \textit{Genetic Algorithm} dan \textit{Particle Swarm Optimization}”. 
	Tugas Akhir ini bukan sekadar hasil dari kerja keras dan proses berpikir ilmiah, tetapi juga buah dari dukungan, doa, dan kehadiran banyak pihak yang secara langsung maupun tidak langsung telah memberikan kontribusi luar biasa. Oleh karena itu, dengan penuh hormat dan rasa terima kasih yang mendalam, penulis ingin menyampaikan apresiasi kepada:
	\begin{enumerate}[itemsep=1pt, topsep=0pt, parsep=0pt, partopsep=0pt]
		\item {Tuhan Yesus Kristus yang telah memberikan penulis kekuatan, kesehatan, dan hikmat dalam proses penyelesaian Tugas Akhir ini, tanpa kasih karunia-Nya, penulis tidak akan mampu menyelesaikannya ini.}
		\item {Ibu Sipta Purba dan Bapak Sempakata Bangun selaku Orang Tua  tercinta yang menjadi sumber kekuatan penulis dengan memberikan cinta kasih, dukungan, doa, dan harapan kepada penulis dalam menyelesaikan Tugas Akhir ini. Sehingga Tugas Akhir ini dapat terselesaikan dengan baik dan penulis persembahakan sebagai gelar yang akan disandang penulis.}
		\item {Pihak PT Perkebunan Nusantara IV Regional 7 Kebun Bekri, yang telah memberikan akses, izin, serta arahan selama pelaksanaan penelitian lapangan untuk Tugas Akhir ini dapat berjalan dengan lancar.}
		\item {Bapak Andika Setiawan, S.Kom., M.Cs., Koordinator Program Studi Teknik Informatika ITERA sekaligus Dosen Penguji 2, atas bimbingan, kritik yang membangun, dan kepercayaan yang diberikan hingga Tugas Akhir ini dapat terselesaikan. Penulis juga berterima kasih atas kesempatan yang diberikan untuk mengikuti Magang Prodi dan Laboran 2025 di Prodi Teknik Informatika.}
		\item {Bapak Meida Cahyo Untoro, S.Kom. M.Kom. selaku Koordinator TA dan Dosen Pembimbing yang telah memberikan masukan, saran dan kelancaran kepada penulis dalam menyelesaikan Tugas akhir ini.}
		\item {Bapak Martin Clinton Tosima Manullang, Ph.D. selaku Dosen Penguji 1, atas arahan, ilmu, serta kesempatan berharga yang diberikan dalam mempertajam kemampuan ilmiah, memperdalam diskusi, dan memperluas wawasan riset yang sangat memperkaya proses penyelesaian Tugas Akhir ini.}
		\item {Bapak Andre Febrianto, S.Kom., M.Eng., selaku Dosen Pembimbing Akademik (Dosen Wali) yang mendampingi penulis sejak awal perkuliahan hingga akhir studi, atas arahan dan dukungan yang sangat membantu penyelesaian studi di Program Studi Teknik Informatika ITERA.}
		\item {Rekan-rekan Capstone Project, yaitu Benedictus Budhi Dharmawan dan Tobyanto Putra Mandiri, atas kebersamaan, dukungan, dan pemikiran berharga yang senantiasa dibagikan sepanjang perjalanan ini, sehingga setiap tantangan terasa lebih ringan.}
		\item {Teman-teman kecil saya, yaitu Emmia Sindilosa Ginting, Chetrine br Milala, dan Brigita May Putri Rosari Sidabutar, yang hadir dalam suka dan duka. Terima kasih telah menjadi tempat berbagi cerita, bertumbuh bersama, dan menguatkan penulis hingga akhir masa studi dan proses penyelesaian Tugas Akhir ini.}
		\item {Rekan-rekan yang memberi warna, semangat, dan keceriaan selama perjalanan Tugas Akhir ini, yaitu Steganonegai Team (Alvin-Marchell-Attar), Event Pejuang TA (Alvin, Atha, Attar, Marchell, Nopri, Pandu, Tara, Qaessar, dan Vania), FamilyMart Team, Stoberi Salkomsel, Proyek Kelapa Sawit Team, Ruang TA Team, Permata GBKP Runggun Bandar Lampung, dan BOS Team. Terima kasih atas setiap tawa, dukungan, kebersamaan yang menjadi penguat luar biasa, kehadiran kalian benar-benar berarti bagi penulis.}
		\item {Bapak Radhinka Bagaskara, S.Si.Kom., M.Si., M.Sc., selaku Koordinator Laboratorium Teknik 3 dan pengelola Ruang TA, atas fasilitas dan suasana kerja yang kondusif yang diberikan kepada penulis dalam menyelesaikan riset ini, bahkan hingga larut malam. Ucapan terima kasih juga penulis sampaikan kepada Ibu Leslie Anggraini, S.Kom., M.Cs. dan Ibu Miranti Verdiana, M.Si., atas dukungan, semangat, serta perhatian tulus yang senantiasa diberikan selama masa studi penulis di Program Studi Teknik Informatika ITERA.}
		\item {Seluruh Dosen, Tenaga Pendidik, dan Staff Program Studi Teknik Informatika ITERA yang telah memberikan ilmu, inspirasi dan pengalaman selama perjalanan akademik penulis.}
		\item {Seluruh teman seperjuangan Teknik Informatika ITERA, khususnya Angkatan 2021 (Binary) atas kebersamaan dan dukungannya selama masa studi penulis.}
		\item {Diri penulis sendiri, atas dedikasi yang telah ditunjukkan selama proses perkuliahan hingga penyusunan Tugas Akhir, dan kesabaran dalam menghadapi berbagai tantangan yang muncul.}
		\item {Untuk seluruh pihak yang tidak dapat disebutkan satu per satu, terima kasih atas segala bentuk dukungan, baik langsung maupun tidak langsung yang telah menguatkan penulis selama menempuh studi dan menyelesaikan Tugas Akhir ini.}
	\end{enumerate} \par
	Akhir kata, penulis berharap semoga tugas akhir ini tidak hanya menjadi pencapaian pribadi, tetapi juga memberikan kontribusi nyata bagi pengembangan ilmu pengetahuan, khususnya dalam penerapan \textit{Machine Learning} untuk sektor pertanian di Indonesia. Amin.
	\vfill
	
\end{justifying}
\clearpage
