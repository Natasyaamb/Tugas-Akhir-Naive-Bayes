\clearpage
\phantomsection% 
\addcontentsline{toc}{chapter}{Abstract}
\thispagestyle{fancy}

\begin{center}
	\large \bfseries \MakeUppercase{Abstract}\\
	\normalsize \normalfont {\thetitleEN}\\
	\normalsize \normalfont {\theauthor}\\
	\bigskip
	
	\normalsize \normalfont \justifying \singlespacing
	Palm oil is a leading commodity in Indonesia with a significant role in the national economy. However, during the growth phase, palm oil seedlings are vulnerable to various diseases, particulary on the leaves, which can reduce seedling quality and productivity. Therefore, early detection of leaf diseases is essential to ensure seedling quality. This study aims to develop a classification model for leaf diseases detection in palm oil seedlings based on digital image processing and machine learning. Texture and RGB features were extracted using the Gray Level Co-occurremce Matrix (GLCM) method. Theses features were then classified using the Naïve Bayes algorithm, optimized with 2 techniques: Genetic Algorithm (GA) and Particle Swarm Optimization (PSO), to improve the model's training accuracy. The dataset used includes 5 classes of diseases. Experimental results show that the best model achieved an accuracy of 56\%, with average precision, recall, and f1-score values ranging from 55\% to 56\%. This approach demonstrates good computational efficiency and can serve as a baseline for the development of more accurate classification models in the future.
	
	\textbf{Keywords: Disease Identification, Palm oil, Naive Bayes, Genetic Algorithm, Particle Swarm Optimization}
	
	\vfill
	
\end{center}
\clearpage