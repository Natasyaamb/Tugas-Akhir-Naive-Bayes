\clearpage
\phantomsection% 
\addcontentsline{toc}{chapter}{Abstrak}
\thispagestyle{fancy}

\begin{center}
	\large \bfseries \MakeUppercase{Abstrak}\\
	\normalsize \normalfont {\thetitle}\\
	\normalsize \normalfont {\theauthor}\\
	\bigskip
	
	\normalsize \normalfont \justifying \singlespacing
	Tanaman kelapa sawit merupakan komoditas unggulan Indonesia yang memiliki peran penting dalam perekonomian nasional. Namun pada fase pertumbuhan, bibit kelapa sawit rentan terhadap  serangan berbagai penyakit, terutama pada daun yang dapat menurunkan kualitas dan produktivitas bibit. Oleh karena itu, deteksi dini terhadap penyakit berperan penting dalam menjamin kualitas bibit. Penelitian ini bertujuan untuk mengembangkan model klasifikasi penyakit pada daun bibit kelapa sawit berbasi pengolahan citra digital dan machine learning. Ekstraksi fitur tekstur dan nilai RGB dari daun dilakukan dengan metode Gray Level Co-occurrence Matrix (GLCM). Fitur tersebut kemudian diklasifikasi dengan algoritma Naïve Bayes yang dioptimasi dengan dua metode, yakni Genetic Algorithm (GA) dan Particle Swarm Optimization (PSO) untuk meningkatkan akurasi pelatihan dari model. Dataset yang digunakan mencakup 5 kelas penyakit. Hasil pengujian menunjukkan bahwa model terbaik menghasilkan akurasi sebesar 56\% dengan rata-rata nilai presisi, recall, dan f1-score berkisar antara 55\% hingga 56\%. Pendekatan ini menunjukkan efisiensi komputasi yang baik dan dapat dijadikan baseline dalam pengembangan model klasifikasi yang lebih akurat di masa mendatang.

	
	\textbf{Kata Kunci:  Identifikasi Penyakit, Kelapa Sawit, Naïve Bayes, Genetic Algorithm, Particle Swarm Optimization
}
	
	\vfill
	
\end{center}
\clearpage